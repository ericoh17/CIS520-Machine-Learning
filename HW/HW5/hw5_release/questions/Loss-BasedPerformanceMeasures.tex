\section{Loss-Based Performance Measures\hpoints{25}}
%
\begin{enumerate}
\item
\textbf{Binary classification with asymmetric costs.\hpoints{5}}

Consider a binary classification problem in which the cost of a false positive is 10 and that of a false negative is 40, so that we have the following loss:
%
\vspace{-12pt}
\begin{eqnarray*}
& \rule{0.75cm}{0pt} \hat{y} 
\\
y &
\begin{tabular}{@{}c|cc}
 & $-1$ & $+1$ \\
 \hline
$-1$ & $0$ & $10$ \\[4pt]
$+1$ & $40$ & $0$ \\
\end{tabular}
\end{eqnarray*}
%
Say you have learned a class probability estimation (CPE) model $\hat{\eta}:\X\>[0,1]$, which given an instance $x\in\X$, estimates the probability of $x$ having a true label $+1$.
How would you use this to build a classification model $h:\X\>\{-1,+1\}$ for the above problem? Give your answer as a decision rule that explains when $h(x)$ should be $-1$ or $+1$ (as a function of $\hat{\eta}(x)$).
\item
\textbf{Binary classification with abstain option.\hpoints{10}}

Consider a binary classification problem in which the classifier is allowed to `abstain' on instances it is not sure about (such instances could then be sent to a human expert to classify, with some associated cost). So here the true label $y$ can take 2 possible values, $-1$ and $+1$, while the predicted label $\hat{y}$ can take 3 possible values: $-1$, $+1$, and `?' (`abstain'). Suppose that the cost of misclassifying an instance is 1 as in the usual 0-1 loss, and that the cost of abstaining is $0.4$, so that we have the following loss:
%
%\vspace{-12pt}
\begin{eqnarray*}
& \rule{0.75cm}{0pt} \hat{y} 
\\
y &
\begin{tabular}{@{}c|ccc}
 & $-1$ & $+1$ & `?' \\
 \hline
$-1$ & $0$ & $1$ & $0.4$ \\[4pt]
$+1$ & $1$ & $0$ & $0.4$ \\
\end{tabular}
\end{eqnarray*}
%
Say you have learned a class probability estimation (CPE) model $\hat{\eta}:\X\>[0,1]$, which given an instance $x\in\X$, estimates the probability of $x$ having a true label $+1$.
How would you use this to build a classification model $h:\X\>\{-1,+1,\text{`?'} \}$ for the above problem? Give your answer as a decision rule that explains when $h(x)$ should be $-1$, $+1$, or `?' (as a function of $\hat{\eta}(x)$). Explain your derivation. 
If the cost of abstaining changes from 0.4 to 0.2, how will your classification model change? Will it abstain more frequently or less frequently?
%
\item
\textbf{Multiclass classification.\hpoints{10}}

You are collaborating with a cancer treatment center and are trying to help them predict which patients will respond well to a particular cancer drug.
You are given clinical data for patients they have given the drug to in the past; for each such patient, the data contains measurements from the patient's tumor biopsy together with a class label indicating whether the patient was a complete responder (CR) to the drug, a partial responder (PR), or a non-responder (NR).
Your goal is to predict the response category (CR, PR, or NR) for new patients based on their tumor biopsy measurements.
You are given the following loss for this problem:
%
\begin{center}
\begin{tabular}{cc|ccc}
\multicolumn{3}{c}{} & \multicolumn{2}{l}{$\hat{y}$} \\[2pt]
 & & NR & PR & CR \\
 \cline{2-5}
 & NR & $0$ & $4$ & $5$ \\
$y$ & PR & $9$ & $0$ & $1$ \\
 & CR & $10$ & $1$ & $0$ \\
\end{tabular}
\end{center}
%
Here mis-predicting a PR case as CR or vice versa incurs relatively little cost, since in both cases the patient is given the drug. Mis-predicting a PR or CR case as NR is very costly, since in this case a patient who could benefit from the drug does not receive treatment. Mis-predicting an NR case as PR or CR is also costly, since it involves unnecessary expense and side effects for a patient who doesn't benefit from the drug, but is less costly than errors in the other direction. 

Suppose that, using the training data provided to you, you have trained a CPE model, and that for 2 new patients with measurement vectors $x_1$ and $x_2$, the model estimates the following probabilities for the 3 classes:
%
%\begin{eqnarray*}
\[
\text{Patient 1:}~~ 
\left( \begin{array}{l}
	\hat{\eta}_{\text{NR}}(x_1) \\
	\hat{\eta}_{\text{PR}}(x_1) \\
	\hat{\eta}_{\text{CR}}(x_1) 
\end{array} \right)
	= 
\left( \begin{array}{l}
	0.6 \\
	0.3 \\
	0.1 
\end{array} \right)
	\,;
	~~~~
\text{Patient 2:}~~ 
\left( \begin{array}{l}
	\hat{\eta}_{\text{NR}}(x_2) \\
	\hat{\eta}_{\text{PR}}(x_2) \\
	\hat{\eta}_{\text{CR}}(x_2) 
\end{array} \right)
	= 
\left( \begin{array}{l}
	0.1 \\
	0.3 \\
	0.6 
\end{array} \right)
	\,.
\]	
Which response category would you predict for each patient, and how do these categories differ from what you would have predicted under the 0-1 loss? Explain your answers.	
\end{enumerate}
%